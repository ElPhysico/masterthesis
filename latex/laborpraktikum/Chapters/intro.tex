%************************************************
\chapter{Introduction}\label{ch:intro}
%************************************************
What is the common denominator between a human looking for its keys, an animal seeking for food, a T-killer cell on the hunt for cancer cells, and a bacterium trying to find glucose to feast on? No matter how advanced the life form, they are all facing different types of search problems.

As above examples show, search problems are an essential part of life. The efficiency of search strategies -- or in some cases equivalently the time needed to successfully finish a search -- is quite often crucial in survival. Obviously, depending on the life form, different capabilities are available to go about a search problem and thus, different strategies have evolved over the course of evolution. However, as a general prerequisite, all searchers need to be able to move and explore their environment, or more specifically, the search vicinity.

Albeit many reasons for motion are related to search problems and equivalent problems, motion in general is a more essential phenomenon in nature and has been the focus of much research, such as in ancient human migration \cite{flores:2007}, movement of marine species \cite{pittman:2003}, and bacterial migration \cite{harkes:1992}.

Quite often the goal of such research is to identify certain properties of motion or movement patterns. For this purpose, mathematical models provide many useful tools. Especially the theory around random walk models has proven to be of invaluable worth when it comes to modeling random motion.

However, the main interest of this work lies in search problems and related search strategies. With motion models provided by the random walk theory, the next step is to analyze such random search problems. Therefore, the first-passage theory serves as an adequate tool to calculate search times and compare different search strategies.

In addition, random search problems can be expanded with hints in the environment that help the searcher find the target. This corresponds to real biological systems, in which \eg traces, odor, or chemicals are detected by the searcher to obtain information about the whereabouts of the target.

In this work we want to focus on random search problems, which often appear in microscopic systems, \ie for cells and other microorganisms such as bacteria. We will start with an introduction to active motion in biological systems of any size in \autoref{ch:activeMotion}. There, we will also give an overview of related research in different fields. In \autoref{ch:modelingActiveMatter} we will introduce some random walk models and the corresponding mathematical background. Following this, in \autoref{ch:randomSearchStrategies} we will talk about random search strategies as well as the first-passage time theory. We will also briefly review the work of \mycitea{tejedor:2012} on optimizing persistent random walk searches. Finally, we will give a brief introduction of search problems with hints in the environment in \autoref{ch:hints}. For this purpose, among others, we will also take a look at the works of \mycitea{celani:2010} and \mycitea{novikova:2017} on bacterial chemotaxis and durotactic motion of cells.
